% !TEX encoding = UTF-8 Unicode
\documentclass[margin,line]{res}
%\usepackage{helvetica} 
\usepackage{fontspec}
\setmainfont{Helvetica}[
  UprightFont = {*},
  BoldFont = {* Bold},
  SlantedFont = {* Oblique},
  BoldSlantedFont = {* Bold Oblique},
  ItalicFont=Helvetica Neue Italic,
  SmallCapsFont=Helvetica Neue LIght Italic]
%\usepackage{newtxsf}
\usepackage{xcolor}
\usepackage[hidelinks]{hyperref}
\usepackage{wrapfig,graphicx,lipsum}

\oddsidemargin -.7in %how far shifted left/right content is from left-hand side of page
\evensidemargin -.5in % ?
\textwidth=6.5in %how wide the main content (right) column is
\itemsep=0in % ?
\parsep=0in % ?
% TO CHANGE WIDTH OF SECTION HEADERS: open res.cls and change margin in line 445
% TO CHANGE TOP MARGIN OF PAGES: open res.cls and change margin in line 780 (\topmargin)
% TO CHANGE HEIGHT OF CONTENT: open res.cls and change margin in line 778 (\textheight)

\newenvironment{list1}{
  \begin{list}{$\bullet$}{
      \setlength{\itemsep}{0in}
      \setlength{\parsep}{0in} \setlength{\parskip}{0in}
      \setlength{\topsep}{0in} \setlength{\partopsep}{0in} 
      \setlength{\leftmargin}{0.17in}}}{\end{list}} % ?

\begin{document}

\name{\vspace*{-0.28in} \Large Mallory A Freeberg, PhD} %Change to move name up/down

%\begin{figure}[h!]
%    \flushright
%\includegraphics[scale=0.02]{headshot-circle.png} % No idea on how to position this
%\end{figure}

%\setlength{\columnsep}{0pt}
%\begin{wrapfigure}{r}{0cm}
%  \centering
%  \includegraphics[width=0.2\linewidth]{headshot-circle.png}
%\end{wrapfigure}

%\begin{figure}[t!]
%  \includegraphics[width=0.15\linewidth]{headshot-circle.png}
%  \label{fig:boat1}
%\end{figure}

\begin{resume}

\section{\sc Who I Am}
Community Builder. Data Sharing Expert. Bioinformatics \& Metadata Enthusiast. Open Science Advocate.
%Community \& Collaboration Builder | Human Data Sharing Expert | Bioinformatics \& Metadata Enthusiast | Open Science Advocate

\section{\sc Contact}
\begin{tabular}{@{}p{3in}p{3in}} %width of columns
{\it Address:} Wellcome Trust Genome Campus & {\em LinkedIn}: \href{https://www.linkedin.com/in/mallory-freeberg/}{\textcolor{blue}{mallory-freeberg}} \\
%{\it Address:} Wellcome Trust Genome Campus & {\em Twitter}: \href{https://twitter.com/MalloryFreeberg}{\textcolor{blue}{@MalloryFreeberg}} \\
{\em Email:}  mallory.freeberg@gmail.com & {\em ORCID:} \href{https://orcid.org/0000-0003-2949-3921}{\textcolor{blue}{0000-0003-2949-3921}} \\
{} & {\em Website:} \href{https://malloryfreeberg.github.io/whoami/}{\textcolor{blue}{malloryfreeberg.github.io/whoami}} \\
\end{tabular}
% {\em Pronouns:} she/her
%Hinxton, Cambridgeshire, CB10 1SD, UK

%\section{\sc Research Interests}
%Genomics/genetics; post-transcriptional gene regulation; high-throughput technologies; second- and third-generation sequencing; data sharing and reuse; open-source tools and platforms enabling life sciences research

\section{\sc Professional Experience}
{\bf EMBL European Bioinformatics Institute}, Cambridge, UK\\
{\em Coordinator}, European Genome-phenome Archive \hfill {Jun 2021 - present}
\begin{itemize}
\itemsep0em 
	\item Contribute expertise to help establish the Federated EGA Network for transnational discovery of and access to sensitive human data (\href{https://ega-archive.org/blog/safe-access-to-sensitive-human-data-across-borders-federated-ega/}{\textcolor{blue}{blog post}})
	\item Engage with external partners and stakeholders to ensure data submission, discovery, and distribution services align with community standards
	\item Manage up to14-person team responsible for operating production data archive services (\href{https://doi.org/10.1093/nar/gkab1059}{\textcolor{blue}{publication}})
\end{itemize}

{\em Project Lead, UK Biobank}, European Genome-phenome Archive \hfill {Aug 2019 - May 2021}
\begin{itemize}
\itemsep0em 
	\item Oversaw data flow to archive 500,000 whole human genomes totalling 12PB of data
	\item Built and led 3-person team to successfully distribute 4PB of data to cloud-based analysis platform
\end{itemize}

{\em Senior Bioinformatician}, Human Cell Atlas Data Coordination Platform \hfill {Oct 2017 - Jul 2019}
\begin{itemize}
\itemsep0em 
	\item Coordinated data wrangling work for 6-person team across UK- and US-based institutions
	\item Performed curation and ingestion of flagship HCA datasets into the Platform
	\item Developed HCA metadata standard for cellular-resolution transcriptomics data (\href{https://doi.org/10.1038/s41587-020-00744-z}{\textcolor{blue}{publication}})
	%\item Represented HCA DCP by presenting at three international HCA meetings
\end{itemize}

%{\em Bioinformatician}, Human Cell Atlas Data Coordination Platform \hfill {Oct 2017 - Sep 2018}
%\begin{itemize}
%\itemsep0em 
%	\item Performed curation and ingestion of three flagship HCA datasets
%	\item Developed HCA metadata standard for cellular-resolution transcriptomics data (\href{https://doi.org/10.1038/s41587-020-00744-z}{publication})
%\end{itemize}

{\bf Johns Hopkins University}, Baltimore, MD USA\\
%{\bf The Galaxy Project}, Baltimore, MD USA\\
{\em Trainer}, Galaxy Training Network, The Galaxy Project \hfill {Jan 2017 - Sep 2017}
\begin{itemize}
\itemsep0em 
	\item Designed and delivered seven Galaxy training workshops in three countries (\href{https://doi.org/10.1016/j.cels.2018.05.012}{\textcolor{blue}{publication}})
	\item Created training materials and computational workflows to support reproducible transcriptomics research (\href{https://training.galaxyproject.org/training-material/hall-of-fame/malloryfreeberg/}{\textcolor{blue}{contributions}})
\end{itemize}

{\em Postdoctoral Research Fellow}, Department of Biology \hfill {Jun 2015 - Sep 2017}
\begin{itemize}
\itemsep0em 
	\item Investigated post-transcriptional gene regulatory networks using neural networks
	\item Piloted Oxford Nanopore direct RNA sequencing protocols to catalog transcriptomes
\end{itemize}

%Advisor:  Dr. James Taylor

%{\bf University of Michigan}, Ann Arbor, MI USA\\
%{\em Graduate Student} \hfill {Jul 2009 - May 2015}\\
%Dissertation: Computational analysis of the post-transcriptional gene regulatory network\\
%Advisor:  Dr. John K. Kim

%{\bf Saint Vincent College}, Latrobe, PA USA\\
%{\em Undergraduate Student} \hfill {Jan 2008 - May 2009}\\
%Dissertation: Functional annotation of non-coding elements in the Amphioxus genome\\
%Advisor:  Dr. Michael L. Sierk\\

%{\bf Boyce Thompson Institute for Plant Research}, Ithaca, NY USA\\
%{\em Summer Research Intern} \hfill {May - Aug 2008}\\
%Research: Designed and implemented a motif discovery tool to identify conserved DNA motifs in untranslated gene regions from plants in the {\em Solanaceae} family\\
%Research: Developed computational tool to discover conserved motifs in {\em Solanaceae} genomes\\
%Advisor: Dr. Lukas A. Mueller

%{\bf Johns Hopkins University Applied Physics Laboratory}, Laurel, MD USA\\
%{\em Summer Research Intern} \hfill {May - Aug 2007}\\
%Research: Developed pipeline and built database to identify microorganisms in an environmental sample by comparing mass spectrometry spectral peak data to known protein molecular weights\\
%Research: Designed computational pipeline for microorganism identification by mass spectrometry\\
%Advisors: Drs. Plamen A. Demirev and Richard S. Potember

%{\bf Saint Vincent College}, Latrobe, PA USA\\
%{\em Summer Research Intern} \hfill {May - Aug 2006}\\
%Evaluated the effectiveness of protein databases to classify protein domains using pairwise-, profile-, and structure-based alignment algorithms\\
%Advisor: Dr. Michael L. Sierk

\section{\sc Education}
{\bf PhD Bioinformatics}, University of Michigan, USA \hfill {May 2015}\\
Department of Computational Medicine and Bioinformatics\\ 
Dissertation: Computational analysis of the post-transcriptional gene regulatory network (\href{https://deepblue.lib.umich.edu/bitstream/handle/2027.42/111339/mafree_1.pdf?sequence=1&isAllowed=y}{\textcolor{blue}{thesis}})

{\bf BSc Bioinformatics}, Saint Vincent College, USA \hfill {May 2009}\\
Herbert M. Boyer School of Natural Science, Mathematics, and Computing\\
Dissertation: Functional annotation of non-coding elements in the Amphioxus genome
%{Minors: Biochemistry, Computing/Information Sciences}

\section{\sc Selected Publications}

Corvo A, Matalonga L, Spalding D, {\em et al.} (2023) Remote visualization of large-scale genomic alignments for collaborative clinical research and diagnosis of rare diseases. {\em Cell Genomics}. DOI:\href{https://doi.org/10.1016/j.xgen.2022.100246}{\textcolor{blue}{10.1016/j.xgen.2022.100246}}

Thakur M, Bateman A, Brooksbank C, Freeberg MA, {\em et al.} (2022) EMBL's European Bioinformatics Institute (EMBL-EBI) in 2022. {\em Nucleic Acids Res}. DOI:\href{https://doi.org/10.1093/nar/gkac1098}{\textcolor{blue}{10.1093/nar/gkac1098}}

Freeberg MA, Fromont L, D'Altri T, {\em et al.} (2021) The European Genome-phenome Archive in 2021. {\em Nucleic Acids Res}. DOI:\href{https://doi.org/10.1093/nar/gkab1059}{\textcolor{blue}{10.1093/nar/gkab1059}}

Rehm HL, Page AJH, Smith L, {\em et al.} (2021) GA4GH: International policies and standards for data sharing across genomic research and healthcare. {\em Cell Genomics} 1. DOI:\href{https://doi.org/10.1016/j.xgen.2021.100029}{\textcolor{blue}{10.1016/j.xgen.2021.100029}}

%Lawson J, Cabili MN, Kerry G, {\em et al.} (2021) The Data Use Ontology to streamline responsible access to human biomedical datasets. {\em Cell Genomics}.\\ DOI:\href{https://doi.org/10.1016/j.xgen.2021.100028}{\textcolor{blue}{10.1016/j.xgen.2021.100028}}

Füllgrabe A, George N, Green M, {\em et al.} (2020) Guidelines for reporting single-cell RNA-seq experiments. {\em Nat Biotechnol} 38:1384-6. DOI:\href{https://www.nature.com/articles/s41587-020-00744-z}{\textcolor{blue}{10.1038/s41587-020-00744-z}} ({\em arXiv} preprint DOI:\href{https://arxiv.org/abs/1910.14623}{{1910.14623v1}})

%Batut B, Hiltemann S, {\em et al.} (2018) Community-Driven Data Analysis Training for Biology. {\em Cell Syst} 6(6):752-758. DOI:\href{https://doi.org/10.1016/j.cels.2018.05.012}{\textcolor{blue}{10.1016/j.cels.2018.05.012}} ({\em bioRxiv} preprint DOI:\href{https://www.biorxiv.org/content/10.1101/225680v2}{{10.1101/225680}})

\Rightarrow  Full list of publications can be found on \href{https://scholar.google.com/citations?user=2LCcJA0AAAAJ&hl=en}{\textcolor{blue}{Google Scholar}}.

\section{\sc Selected\\ Invited Talks}
%"Title", {\em Conference}; MMM YYYY; City, State/Country (materials)

Federated EGA: Providing global discovery and access for sensitive human data. {\em ISMB/ECCB}; Jul 2023; Lyon, FR (\href{https://docs.google.com/presentation/d/1P9KMd-NAjbz1f9fgO9FVuyYTVAoYZoJAqL0wvnVNrt4/edit?usp=sharing}{\textcolor{blue}{slides}})

The Federated EGA: Global discovery and access for sensitive human data. {\em GA4GH 10th Plenary}; Sep 2022; Barcelona, ES (\href{https://docs.google.com/presentation/d/17wJu5ntPdT1Uj3kSOd_ROHY0DK7IzRwRq6Lgyvso2aw/edit?usp=sharing}{\textcolor{blue}{slides}})

%Sharing sensitive human genomic, phenotypic, and clinical data. {\em Bioinformatics for BioBusiness}; Jul 2022; Manchester, UK (\href{https://docs.google.com/presentation/d/16bB4JdxSbHiNxIqIkVvqjt7rUH4oRMDivPrZRNFp27c/edit?usp=sharing}{\textcolor{blue}{slides}})

FAIR principles and promoting openness in the life sciences. {\em NIHR Statistics Group Workshop}; Feb 2022; virtual (\href{https://docs.google.com/presentation/d/1BNdhj_Ny7qJ84xmoUuzqIicYXF3UwwcGdJTLtyZkV5I/edit?usp=sharing}{\textcolor{blue}{slides}})

The European Genome-phenome Archive Metadata Model. {\em GHGA Workshop on Metadata in Biomedical Genome Research}; Jun 2021; virtual (\href{https://docs.google.com/presentation/d/14x9PNs3j5mNZWIHBW9k6aiJUekiLYsrbrkpaMihB4ko/edit?usp=sharing}{\textcolor{blue}{slides}})

%Human Cell Atlas Data Coordination Platform. {\em Pilot Projects for a Human Cell Atlas Europe Retreat}; Aug 2018; Stockholm, Sweden

%Human Cell Atlas: Google Maps for human anatomy. Cambridge UK Computational Biology and Bioinformatics Meetup; Jun 2018; Cambridge, UK

%Human Cell Atlas Data Coordination Platform Status Update. {\em Human Cell Atlas General Meeting}; Mar 2018; Hinxton, UK (\href{https://youtu.be/NrYH9Oc6Cyk}{\textcolor{blue}{video}})

%Single molecule RNA sequencing of the {\em C. elegans} transcriptome. {\em Molecular Biosystems Conference on Eukaryotic Gene Regulation and Functional Genomics}; Sep 2017; Puerto Varas, CL (\href{https://drive.google.com/file/d/1PKdFgswUhp4K3S_j61xXH5695tnv0Kty/view?usp=sharing}{\textcolor{blue}{slides}})\\
%{\sc Sponsored by Oxford Nanopore Technologies}

%From high school to postdoc: lessons from a decade of bioinformatics education. Great Lakes Bioinformatics Conference; May 2017; University of Illinois at Chicago, Chicago, IL. DOI:10.7490/f1000research.1114161.1

%A novel pipeline for identifying transcriptome-wide binding sites of RNA-binding proteins from PAR-CLIP sequencing data. NCIBI Tools and Technology Seminar Series; May 2013; University of Michigan, Ann Arbor, MI

%Recommendations for successful scientific data sharing: Case studies of next-generation sequencing data use and re-use. NSF Open Data IGERT Seminar; Mar 2013; University of Michigan, Ann Arbor, MI\\

%Defending the genome: Computational identification of germline-specific piRNAs in {\em C. elegans}. Sep 2012; Saint Vincent College, Latrobe, PA

%Next-generation sequencing data re-use facilitates studies of small RNAs. NSF Open Data IGERT Research Workshop; Apr 2012; University of Michigan, Ann Arbor, MI\\

%\Rightarrow  Full list of talks can be found on my \href{https://malloryfreeberg.github.io/whoami/}{\textcolor{blue}{website}}.

\section{\sc Selected\\ Conferences\\ \& Workshops}

%"Title", {\em Conference}; MMM YYYY; City, State/Country (materials)

Demonstrating federated EGA services for sensitive data discovery and access. {\em ELIXIR All Hands}; Jun 2023; Dublin, IE (\href{https://f1000research.com/slides/12-638}{\textcolor{blue}{workshop}})

Resource considerations for establishing and operating a federated human data node. {\em ELIXIR All Hands}; Jun 2022; Amsterdam, NL (\href{https://f1000research.com/posters/11-594}{\textcolor{blue}{poster}})

%Federated EGA Updates in 2022. {\em ELIXIR All Hands 2022}; Jun 2022; Amsterdam, NL (\href{https://f1000research.com/slides/11-623}{\textcolor{blue}{workshop}})

Managing single cell transcriptomics data. {\em EMBL-EBI Training Course}; Jul 2019; Cambridge, UK (\href{https://www.ebi.ac.uk/training/events/managing-single-cell-transcriptomics-data/}{\textcolor{blue}{course}})

%Engaging with the Data Coordination Platform. {\em Human Cell Atlas General Meeting}; May 2019; Tokyo, JP

The Data Coordination Platform: Making the Human Cell Atlas data easily accessible. {\em The Biology of Genomes}; May 2018; Cold Spring Harbor, NY USA (\href{https://drive.google.com/file/d/1KMs4VWE_1rEAFeKLHWrFxHaly5Db78By/view?usp=sharing}{\textcolor{blue}{poster}})

%Approaches for small RNA-seq in Galaxy.{\em Galaxy Community Conference}; Jun 2017; Montpellier, FR (\href{https://doi.org/10.7490/f1000research.1114423.1}{\textcolor{blue}{slides}})

%{\bf Freeberg MA} and Heydarian M. Galaxy 101: A gentle introduction to Galaxy. Galaxy Community Conference; Jun 2017; Montpellier, France (Workshop)

%{\bf Freeberg MA} and Heydarian M. RNAseq analysis in Galaxy. Galaxy Community Conference; Jun 2017; Montpellier, France (Workshop)

%{\bf Freeberg MA} and Heydarian M. Introduction to Genomic Data Analysis with Galaxy. U. S. Food and Drug Administration; Mar 2017; Silver Spring, MD, USA (Workshop)

%{\bf Freeberg MA} and Heydarian M. Introduction to Genomic Data Analysis with Galaxy. Global Biodiversity Genomics Conference; Feb 2017; Washington D.C. (Workshop)

%{\bf Freeberg MA} and Heydarian M. Galaxy to Genomics using NGS Data. 13th KOGO Winter Symposium; Feb 2017; Hongcheon, South Korea (Workshop)

%{\bf Freeberg MA} and Heydarian M. Galaxy to Genomics using NGS Data. Feb 2017; Daejeon and Seoul, South Korea (Workshops)

%{\bf Freeberg MA} and Turaga N. Bioinformagic: Marrying Bioconductor and Galaxy. Galaxy Community Conference; Jun 2016; Bloomington, IN, USA (Workshop)

%{\bf Freeberg MA}. Transcriptome sequencing in nematodes. Oxford Nanopore Technologies London Calling; May 2016; London, UK (Poster)

%{\bf Freeberg MA} and Taylor J. Probabilistic modeling of protein:RNA interaction data identifies functional Transcript States. CSHL Genome Informatics; Oct 2015; Cold Spring Harbor, NY (Poster)

%{\bf Freeberg MA}. Identification and characterization of the mRNA-binding proteome {\em in vivo} in {\em Saccharomyces cerevisiae}. CSHL Systems Biology: Global Regulation of Gene Expression; Mar 2014; Cold Spring Harbor, NY (Talk)

% {\bf Freeberg MA}, Billi AC, and Kim JK. Germline expression, inheritance, and genomic characteristics of {\em Caenorhabditis elegans} 21U-RNAs. 18th International {\em C. elegans} Meeting; Jun 2011; Los Angeles, CA (Poster)

% {\bf Freeberg MA}. Functional Annotation of Non-Coding Elements in the Amphioxus Genome. Saint Vincent College 6th Annual Undergraduate Conference; Apr 2009; Latrobe, PA. (Talk)\\

% {\bf Freeberg MA} and Demirev PA. Bioinformatics Approach to Protein Database Generation for the Identification of Microorganisms by Mass Spectrometry. Saint Vincent College 5th Annual Undergraduate Conference; Apr 2008; Latrobe, PA. (Poster)

% {\bf Freeberg MA} and Sierk ML. Assessment of Homolog Detection by Profile-based Protein Sequence Alignment Methods Against a Structure-based Gold Standard. Duquesne University Annual Summer Undergraduate Research Symposium; Aug 2006; Pittsburgh, PA. (Poster)

%\Rightarrow  Full list of conference and workshop materials can be found on my \href{https://malloryfreeberg.github.io/whoami/}{\textcolor{blue}{website}}.

%\section{\sc Funding} 
%XSEDE Educational Computational Resource Allocation \hfill {2017}\\
%XSEDE Startup Computational Resource Allocation \hfill {2017}\\
%{Galaxy Community Fund, Travel Award (2017)}
%{Rackham Conference Travel Grant (2014)}\\
%Rackham Predoctoral Fellowship \hfill {2013-2014}\\
%NSF Open Data IGERT Fellowship \hfill {2013-2014}\\
%Rackham Graduate Student Research Grant \hfill {2010, 2012}\\
%NIH Bioinformatics Training Grant \hfill {2009-2010}

%\vspace{-.2cm}
%{\bf Saint Vincent College}\\
%{Graduated Summa Cum Laude, 2009}\\
%{President's Award Finalist, 2009}\\
%{Outstanding Bioinformatics Student Award, 2009}
%{Who's Who Among Students in American Universities and Colleges   2009}\\
%{Alpha Lambda Delta Academic Honor Society 2006-2009}

%\section{\sc Computing and Data Analysis Skills} 
%Operating Systems: MacOS X, Linux/Unix \\
%Languages: Python, Perl, SQL, Unix shell \\
%Statistical Packages: R, Bioconductor \\
%Applications: JSON/XML, GitHub, \LaTeX, Galaxy, NGS analysis tools and methodologies\\

%Applications: \LaTeX, MySQL, Galaxy, biomolecule sequence/structure analysis tools, NGS analysis tools and methodologies\\
%Laboratory Techniques: Nanopore sequencing, general molecular biology techniques\\
%Laboratory Techniques: Nanopore sequencing (Oxford Nanopore Technologies), general molecular biology techniques

%\section{\sc Volunteer Activities}
%%{Volunteered for FEMMES Capstone Science weekend}\\
%%{Judged Pioneer High School Science Fair, 2014}\\
%%{Judged Forsythe Middle School Science Fair, 2014}\\
%{Peer reviewed for {\em Cell Rep},  {\em Epigenetics}, {\em Genome Biol}, {\em Genome Res}, {\em PLoS Biol}, {\em RNA}, and {\em RNA Biol} (2013-2014)}\\
%%{Judged Webster Elementary School Science Fair and Exhibition, 2013-2014}\\
%%{Organized departmental happy hours following student seminars, 2012-2014}\\
%{Participated in departmental student recruiting activities, University of Michigan (2011-2015)}\\
%%{Judged Elementary, Middle, and High School Science Fairs, 2013-2014}

\section{\sc Leadership Roles}
{\em Co-lead, GA4GH "Future of VCF" Working Group} (Mar 2022 - Present)\\
Guide discussions and assessment of solutions for scaling the Variant Call File format for representing genetic variants across large-scale studies

{\em Organiser, ELIXIR Federated Human Data Community} (Feb 2022 - Present)\\
Set agenda and priority topics for FHD Community meetings, co-organise FHD Community Day event, coordinate federated human data activities across ELIXIR members

{\em Chair, Federated EGA Operations Committee} (Jan 2021 - Present)\\
Lead Committee in fulfilment of responsibilities, coordinate with Committee members to make progress towards shared goals and milestones

\section{\sc Teaching \& Mentoring}
{\em Mentor, general} (2020 - Present)\\
Mentor staff and peers to support them achieving their career and professional development goals

{\em Mentor, \href{https://openlifesci.org/}{\textcolor{blue}{Open Life Science}}} (Jul 2020 - Present)\\
Mentor participants on open science practices as part of a 16-week mentorship and training program to empower open science ambassadors

{\em \href{https://www.stem.org.uk//}{\textcolor{blue}{UK STEM Ambassador}}} (Jan 2018 - Present)\\
Give talks, demonstrate science experiments, share STEM career guidance

%{\em Instructor, Johns Hopkins University}, Baltimore, MD USA\hfill {Jan 2017 - May 2017}\\
%Create materials for the Advanced Academic Program online course\\
%"Bioinformatics: Tools for Genome Analysis", hold office hours, grade assignments

%{\bf Sisters Circle}, Baltimore, MD USA\hfill {May 2015 - Sep 2017}\\
%Mentor/tutor for middle school girls\\
%Responsibilities: Co-led weekly after-school sessions, tutored math, mentored 1-on-1

%{\bf University of Michigan}, Ann Arbor, MI USA\hfill {Jul 2014 - Aug 2014}\\
%Mentor for high school student\\
%Responsibilities: Designed bioinformatics/genomics research project, taught shell/perl programming

%{\bf Saint Vincent College}, Latrobe, PA USA\\
%Laboratory Assistant for General Chemistry \hfill { Jan - May 2009}\\
%Responsibilities: Held office hours, prepared and supervised laboratory exercises, graded assignments

%Laboratory Assistant for Genes, Cells, and Computers \hfill { Aug - Dec 2008}\\
%Responsibilities: Held office hours, prepared and supervised laboratory exercises, graded assignments

%Tutor for Bioinformatics, Genomics, and Proteomics \hfill { Aug - Dec 2007}\\
%Responsibilities: Held office hours, led study sessions

\section{\sc Training\\ Courses}

ELIXIR LEadership And Diversity mentoring programme \hfill {2023-present}\\
%Storytelling Workshop \hfill {2021}\\
Managing Managers \hfill {2021}\\
%Moving into Management \hfill {2020}\\
Emotional Intelligence and Mindfulness \hfill {2020}\\
Talented Women's Impact Programme \hfill {2019-2020}\\
EICAT Complementary Scientific Skills Training - Scientific Project Management \hfill {2019}\\
Handling Conflict and Difficult Conversations \hfill {2018}

\end{resume}
\end{document}

