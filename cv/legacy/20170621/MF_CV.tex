\documentclass[margin,line]{res}
%\usepackage{helvetica} 
%\usepackage{fontspec}
%\usepackage{newtxsf}
%\setmainfont{Helvetica}

\oddsidemargin -.5in
\evensidemargin -.5in
\textwidth=6in
\itemsep=0in
\parsep=0in

\newenvironment{list1}{
  \begin{list}{$\bullet$}{
      \setlength{\itemsep}{0in}
      \setlength{\parsep}{0in} \setlength{\parskip}{0in}
      \setlength{\topsep}{0in} \setlength{\partopsep}{0in} 
      \setlength{\leftmargin}{0.17in}}}{\end{list}}


\begin{document}
\name{\vspace*{-0.3in} \Large Mallory Ann Freeberg, Ph.D.} %Change to move name up/down

\begin{resume}
\section{\sc Contact Information}
\begin{tabular}{@{}p{2.5in}p{5in}}
3400 N Charles Street & {\it Phone:}  (724) 840-6294 \\            
Mudd Hall 144 & {\it Email:}  mallory dot freeberg at gmail dot com \\ %{\it Fax:}  (734) 615-5493 \\         
%Johns Hopkins University   & {\it Email:} mallory.freeberg@gmail.com \\
Baltimore, MD  21218 USA &  
\end{tabular}

%\section{\sc Research Interests}
%Genomics/proteomics; high-throughput technologies; next-generation sequencing;
%databases as tools for organizing and analyzing large datasets; data sharing and reuse

\section{\sc Education}
{\bf University of Michigan}, Ann Arbor, MI USA\\
{Department of Computational Medicine and Bioinformatics}\\ 
{Ph.D. Bioinformatics, May 2015}

{\bf Saint Vincent College}, Latrobe, PA USA\\
{The Herbet M. Boyer School of Natural Science, Mathematics, and Computing} \\
{B.S. Bioinformatics,  May 2009}
%{Minors: Biochemistry, Computing/Information Sciences}

\section{\sc Research Experience}
{\bf Galaxy Project} and {\bf Galaxy Training Network}, Baltimore, MD USA\\
{\em Team Member} \hfill {Jan 2017 - present}\\
Research: Development of workflows, training materials, and best practices for transcriptomics and genomics-related research

{\bf Johns Hopkins University}, Baltimore, MD USA\\
{\em Postdoctoral Research Fellow} \hfill {Jun 2015 - present}\\
Research: Investigating post-transcriptional gene regulation via next gen and nanopore sequencing\\
Advisor:  Dr. James Taylor

{\bf University of Michigan}, Ann Arbor, MI USA\\
{\em Graduate Student} \hfill {Jul 2009 - May 2015}\\
Dissertation: Computational analysis of the post-transcriptional gene regulatory network\\
Advisor:  Dr. John K. Kim

{\bf Saint Vincent College}, Latrobe, PA USA\\
{\em Undergraduate Student} \hfill {Aug 2005 - May 2009}\\
Dissertation: Functional annotation of non-coding elements in the Amphioxus genome\\
Advisor:  Dr. Michael L. Sierk

{\bf Boyce Thompson Institute for Plant Research}, Ithaca, NY USA\\
{\em Summer Research Intern} \hfill {May - Aug 2008}\\
%Research: Designed and implemented a motif discovery tool to identify conserved DNA motifs in untranslated gene regions from plants in the {\em Solanaceae} family\\
Research: Designed motif discovery tool to identify conserved DNA motifs in {\em Solanaceae} genomes\\
Advisor: Dr. Lukas A. Mueller

{\bf Johns Hopkins University Applied Physics Laboratory}, Laurel, MD USA\\
{\em Summer Research Intern} \hfill {May - Aug 2007}\\
%Research: Developed pipeline and built database to identify microorganisms in an environmental sample by comparing mass spectrometry spectral peak data to known protein molecular weights\\
Research: Developed computational pipeline for microorganism identification by mass spectrometry\\
Advisors: Drs. Plamen A. Demirev and Richard S. Potember

%{\bf Saint Vincent College}, Latrobe, PA USA\\
%{\em Summer Research Intern} \hfill {May - Aug 2006}\\
%Evaluated the effectiveness of protein databases to classify protein domains using pairwise-, profile-, and structure-based alignment algorithms\\
%Advisor: Dr. Michael L. Sierk

\section{\sc Funding and Awards} 
%{\bf University of Michigan}\\
{Extreme Science and Engineering Discovery Environment (XSEDE), Startup Computational Resource Allocation (2017)}\\
{XSEDE, Educational Computational Resource Allocation (2017)}\\
{Galaxy Community Fund, Travel Award (2017)}\\
%{Rackham Conference Travel Grant (2014)}\\
{Rackham Predoctoral Fellowship (2013-2014)}\\
{NSF Open Data IGERT Fellowship (2010-2013)}\\
{Rackham Graduate Student Research Grant (2010, 2012)}\\
{NIH Bioinformatics Training Grant (2009-2010)}

\section{\sc Computing and Data Analysis Skills} 
Operating Systems:  MacOS X, Linux/Unix (limited Windows) \\
Languages: Python, Perl, SQL, Unix shell (limited C++, Java, Javascript, PHP/HTML, CWL)\\
Statistical Packages:  R, Bioconductor (limited MatLab)\\
%Applications: \LaTeX, MySQL, Galaxy, biomolecule sequence/structure analysis tools, NGS analysis tools and methodologies\\
Applications: Galaxy, \LaTeX, GitHub, NGS analysis tools and methodologies, MySQL\\
Laboratory Techniques: Nanopore sequencing (Oxford Nanopore Technologies), general molecular biology techniques

%\vspace{-.2cm}
%{\bf Saint Vincent College}\\
%{Graduated Summa Cum Laude, 2009}\\
%{President's Award Finalist, 2009}\\
%{Outstanding Bioinformatics Student Award, 2009}
%{Who's Who Among Students in American Universities and Colleges   2009}\\
%{Alpha Lambda Delta Academic Honor Society 2006-2009}

\section{\sc Publications}

{\bf Freeberg MA}, Han T, Taylor J, and Kim JK. Transcriptome-wide identification and comparative target analysis of the Pumilio family of RNA-binding proteins in {\em S. cerevisiae}. (in revision)

Jin M*, Fuller GG*, Han T*, Yao Y, Alessi AF, {\bf Freeberg MA}, Roach NP, Moresco JJ, Karnovsky A, Baba M, Yates JR III, Gitler AD, Inoki K, Klionsky DJ, and Kim JK. (2017) Glycolytic Enzymes Coalesce in G Bodies Under Hypoxic Stress. {\em Cell Reports} 20(4):895-908. DOI:10.1016/j.celrep.2017.06.082

Hong S, {\bf Freeberg MA}*, Han T*, Kamath A*, Yao Y, Fukuda T, Suzuki T, Kim JK, and Inoki K. (2017) LARP-1 functions as a molecular switch for mTORC1-mediated translation of an essential class of mRNAs. {\em eLife} 6:e25237. DOI:10.7554/eLife.25237

Weiser NE, Yang DX, Feng S, Kalinava N, Brown KC, Khanikar J, {\bf Freeberg MA}, Snyder MJ, Csankovszki G, Chan RC,  Gu SG, Montgomery TA, Jacobsen SE, and Kim JK. (2017) MORC-1 Integrates Nuclear RNAi and Transgenerational Chromatin Architecture to Promote Germline Immortality. {\em Dev Cell} 41(4):408-23. DOI:10.1016/j.devcel.2017.04.023

Turaga N, {\bf Freeberg MA}, Baker D, Chilton J, The Galaxy Team, Nekrutenko A, and Taylor J. (2016) A guide and best practices for R/Bioconductor tool integration in Galaxy [version 1; referees: 1 approved, 1 approved with reservations]. {\em F1000Research} 5:2757. \\DOI:10.12688/f1000research.9821.1

{\bf Freeberg MA} and Kim JK. (2016) Mapping the Transcriptome-Wide Landscape of RBP Binding Sites using gPAR-CLIP-seq: Bioinformatic Analysis. Invited chapter in {\em Methods in Mol Biol} 1361(6):91-104. DOI:10.1007/978-1-4939-3079-1

Alessi AF*, Khivansara V*, Han T*, {\bf Freeberg MA}, Moresco JJ, Tu PG, Montoye E, Yates JR III, Karp X, and Kim JK. (2015) Casein kinase II promotes target silencing by miRISC through direct phosphorylation of the DEAD-box RNA helicase CGH-1. {\em PNAS} 112(49):E6789. \\DOI:10.1073/pnas.1509499112

Jin M, He D, Backues SK, {\bf Freeberg MA}, Liu X, Kim JK, and Klionsky DJ. (2014) Transcriptional regulation by Pho23 modulates the frequency of autophagosome formation. {\em Curr Biol} 24(12):1314-22. DOI:10.1016/j.cub.2014.04.048

{\bf Freeberg MA}*, Han T*, Moresco JJ, Kong A, Yang Y, Lu Z, Yates JR III, and Kim JK. (2013) Pervasive and dynamic protein binding sites on the mRNA transcriptome in {\em Saccharomyces cerevisiae}. {\em Gen Biol} 14(2):R13. DOI:10.1186/gb-2013-14-2-r13

Billi AC*, {\bf Freeberg MA}*, Day AM, Chun SY, Khivansara V, and Kim JK. (2013) A conserved upstream motif drives autonomous, germline-enriched expression of {\em Caenorhabditis elegans} piRNAs. {\em PLoS Genet} 9(3):e1003392. DOI:10.1371/journal.pgen.1003392\\ Comment in: Burgess DJ. (2013) Small RNAs: Defining piRNA expression. {\em Nat Rev Genet} 14(5):301.

Billi AC, {\bf Freeberg MA}, and Kim JK. (2012) piRNAs and siRNAs collaborate in {\em Caenorhabditis elegans} genome defense. {\em Gen Biol} 13(7):164. DOI:10.1186/gb-2012-13-7-164

Billi AC, Alessi AF, Khivansara V, Han T, {\bf Freeberg M}, Mitani S, and Kim JK. (2012) The {\em Caenorhabditis elegans} HEN1 ortholog, HENN-1, methylates and stabilizes select subclasses of germline small RNAs. {\em PLoS Genet} 8(4):e1002617. DOI:10.1371/journal.pgen.1002617

*equal contribution

%\section{\sc Publications in Preparation}

\section{\sc Conference Presentations and Workshops}
{ \bf Freeberg MA} and Heydarian M. Galaxy 101: A gentle introduction to Galaxy. Galaxy Community Conference; Jun 2017; Montpellier, France. (Workshop)

{ \bf Freeberg MA} and Heydarian M. RNAseq analysis in Galaxy. Galaxy Community Conference; Jun 2017; Montpellier, France. (Workshop)

{ \bf Freeberg MA} and Taylor J. Approaches for small RNA-seq analysis in Galaxy. Galaxy Community Conference; Jun 2017; Montpellier, France. (Talk)

{ \bf Freeberg MA} and Heydarian M. Introduction to Genomic Data Analysis with Galaxy. U. S. Food and Drug Administration; Mar 2017; Silver Spring, MD. (Workshop)

{ \bf Freeberg MA} and Heydarian M. Introduction to Genomic Data Analysis with Galaxy. 13th KOGO Winter Symposium; Feb 2017; Washington D.C. (Workshop)

{ \bf Freeberg MA} and Heydarian M. Galaxy to Genomics using NGS Data. 13th KOGO Winter Symposium; Feb 2017; Hongcheon, South Korea. (Workshop)

%{ \bf Freeberg MA} and Heydarian M. Galaxy to Genomics using NGS Data. Feb 2017; Daejeon and Seoul, South Korea. (Workshops)

{ \bf Freeberg MA} and Turaga N. Bioinformagic: Marrying Bioconductor and Galaxy. Galaxy Community Conference; Jun 2016; Bloomington, IN. (Workshop)

{\bf Freeberg MA}. Transcriptome sequencing in nematodes. Oxford Nanopore Technologies London Calling; May 2016; London, UK. (Poster)

{ \bf Freeberg MA} and Taylor J. Probabilistic modeling of protein:RNA interaction data identifies functional Transcript States. CSHL Genome Informatics; Oct 2015; Cold Spring Harbor, NY. (Poster)

{ \bf Freeberg MA}, Han T, and Kim JK. Identification and characterization of the mRNA-binding proteome {\em in vivo} in {\em Saccharomyces cerevisiae}. CSHL Systems Biology: Global Regulation of Gene Expression; Mar 2014; Cold Spring Harbor, NY. (Talk)

% { \bf Freeberg MA}, Han T, Chun SY, and Kim JK. Systems approaches to study global RNA-protein interactions {\em in vivo} in yeast. Life Sciences Institute Symposium; Nov 2012; Ann Arbor, MI. (Poster)

{\bf Freeberg MA}, Billi AC, and Kim JK. Germline expression, inheritance, and genomic characteristics of {\em Caenorhabditis elegans} 21U-RNAs. 18th International {\em C. elegans} Meeting; Jun 2011; Los Angeles, CA. (Poster)

% {\bf Freeberg MA} and Sierk ML. Functional Annotation of Non-Coding Elements in the Amphioxus Genome. Saint Vincent College 6th Annual Undergraduate Conference; Apr 2009; Latrobe, PA. (Talk)

% {\bf Freeberg MA} and Demirev PA. Bioinformatics Approach to Protein Database Generation for the Identification of Microorganisms by Mass Spectrometry. Saint Vincent College 5th Annual Undergraduate Conference; Apr 2008; Latrobe, PA. (Poster)

% {\bf Freeberg MA} and Sierk ML. Assessment of Homolog Detection by Profile-based Protein Sequence Alignment Methods Against a Structure-based Gold Standard. Duquesne University Annual Summer Undergraduate Research Symposium; Aug 2006; Pittsburgh, PA. (Poster)

\section{\sc Invited Talks}
%{\bf Freeberg MA}. From high school to postdoc: lessons from a decade of bioinformatics education. Great Lakes Bioinformatics Conference; May 2017; University of Illinois at Chicago, Chicago, IL.\\
{\bf Freeberg MA}. From high school to postdoc: Lessons from a decade of bioinformatics education [version 1; not peer reviewed]. {\em F1000Research} 2017, {\bf 6}(ISCB Comm J):801 (slides).\\ DOI: 10.7490/f1000research.1114161.1

{\bf Freeberg MA}. A novel pipeline for identifying transcriptome-wide binding sites of RNA-binding proteins from PAR-CLIP sequencing data. NCIBI Tools and Technology Seminar Series; May 2013; University of Michigan, Ann Arbor, MI.

{\bf Freeberg MA} and Stevenson KR. Recommendations for successful scientific data sharing: Case studies of next-generation sequencing data use and re-use. NSF Open Data IGERT Seminar; Mar 2013; University of Michigan, Ann Arbor, MI.

{\bf Freeberg MA}. Defending the genome: Computational identification of germline-specific piRNAs in {\em C. elegans}. Sep 2012; Saint Vincent College, Latrobe, PA.

{\bf Freeberg MA}. Next-generation sequencing data re-use facilitates studies of small RNAs. NSF Open Data IGERT Research Workshop; Apr 2012; University of Michigan, Ann Arbor, MI.

%\section{\sc University and Community Service}
%%{Volunteered for FEMMES Capstone Science weekend}\\
%%{Judged Pioneer High School Science Fair, 2014}\\
%%{Judged Forsythe Middle School Science Fair, 2014}\\
%{Peer reviewed for {\em Cell Rep},  {\em Epigenetics}, {\em Genome Biol}, {\em Genome Res}, {\em PLoS Biol}, {\em RNA}, and {\em RNA Biol} (2013-2014)}\\
%%{Judged Webster Elementary School Science Fair and Exhibition, 2013-2014}\\
%%{Organized departmental happy hours following student seminars, 2012-2014}\\
%{Participated in departmental student recruiting activities, University of Michigan (2011-2015)}\\
%%{Judged Elementary, Middle, and High School Science Fairs, 2013-2014}

\section{\sc Teaching and Mentoring Experience}
{\bf Johns Hopkins University}, Baltimore, MD USA\\
Instructor for Advanced Academic Program online course\hfill {Jan - May 2017}\\
Course title: {\em Bioinformatics: Tools for Genome Analysis}

%{\bf Sisters Circle}, Baltimore, MD USA\\
%One-on-one mentor for middle school student\hfill { May 2016 - present}

{\bf Sisters Circle}, Baltimore, MD USA\\
Mentor/tutor for middle school girls at Henderson-Hopkins Elementary\hfill {May 2015 - May 2016}\\
Responsibilities: Co-lead weekly after-school sessions, weekly math tutoring

{\bf University of Michigan}, Ann Arbor, MI USA\\
Mentor for a high school student \hfill {Jul - Aug 2014}\\
Responsibilities: Designed bioinformatics/genomics research project, taught shell/perl programming

%{\bf Saint Vincent College}, Latrobe, PA USA\\
%Laboratory Assistant for General Chemistry \hfill { Jan - May 2009}\\
%Responsibilities: Held office hours, prepared and supervised laboratory exercises, graded assignments

%Laboratory Assistant for Genes, Cells, and Computers \hfill { Aug - Dec 2008}\\
%Responsibilities: Held office hours, prepared and supervised laboratory exercises, graded assignments

%Tutor for Bioinformatics, Genomics, and Proteomics \hfill { Aug - Dec 2007}\\
%Responsibilities: Held office hours, led study sessions

\end{resume}
\end{document}

